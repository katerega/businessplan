\documentclass[
a4paper,
twoside,
12pt,
fleqn,
toc=listofnumbered, 
bibliography=totocnumbered]{scrartcl}

% table spaces
\usepackage{array}
\renewcommand{\arraystretch}{1.5}

% deutsche Trennmuster etc.
\usepackage[T1]{fontenc}
\usepackage[utf8]{inputenc}
\usepackage[english, ngerman]{babel} % \selectlanguage{english} if  needed
\usepackage{lmodern} % use modern latin fonts

% Custom commands
\newcommand{\AUTHOR}{Michael Wieland}
\newcommand{\SECONDAUTHOR}{Fabian Hauser}
\newcommand{\THIRDAUTHOR}{Murièle Trentini}
\newcommand{\FOURTHAUTHOR}{Patrik Scherler}
\newcommand{\INSTITUTE}{Hochschule für Technik Rapperswil}
\newcommand{\LECTURER}{Prof. Dr. Jürg Stadelwieser}
\newcommand{\GITHUB}{https://github.com/michiwieland/businessplan}

% Jede Überschrift 1 auf neuer Seite
\let\stdsection\section
\renewcommand\section{\clearpage\stdsection}

% Multiple Authors
\usepackage{authblk}

% Include external pdf
\usepackage{pdfpages}

% Layout / Seitenränder
%\usepackage{geometry}
\usepackage[hcentering,bindingoffset=8mm]{geometry}

% Inhaltsverzeichnis
\usepackage{makeidx} 
\makeindex

\usepackage{url}
\usepackage[pdfborder={0 0 0}]{hyperref}
\usepackage[all]{hypcap}
\usepackage{hyperxmp} % for license metadata
\urlstyle{same}

% Glossar und Abkürzungsverzeichnis
\usepackage[acronym,toc,nopostdot]{glossaries}
\glossarystyle{altlist}
\usepackage{xparse}
\DeclareDocumentCommand{\newdualentry}{ O{} O{} m m m m } {
	\newglossaryentry{gls-#3}{
		name={#4 : #5},
		text={#5\glsadd{#3}},
		description={#6},
		#1
	}
	\makeglossaries
	\newacronym[see={[Siehe:]{gls-#3}},#2]{#3}{#4}{#5\glsadd{gls-#3}}
}
\makeglossaries
\loadglsentries{glossar}

% Mathematik
\usepackage{amsmath}
\usepackage{amssymb}
\usepackage{amsfonts}
\usepackage{enumitem}

% Images
\usepackage{graphicx}
\graphicspath{{images/}} % default paths

%figures
\usepackage{tikz}
\usetikzlibrary{shapes.geometric}

% Boxes
\usepackage{fancybox}

%Tables
\usepackage{tabu}
\usepackage{booktabs} % toprule, midrule, bottomrule
\usepackage{array} % for matrix tables

% Multi Columns
\usepackage{multicol}

% Header and footer
\setkomafont{pagehead}{\normalfont\sffamily}
\usepackage[automark]{scrlayer-scrpage}
\automark*{section}
\clearpairofpagestyles
\ihead{\headmark}
\ohead{\TITLE}
\cfoot{\pagemark}
\setlength{\headsep}{30pt}


%Defining the Layer
\makeatletter
\newlength{\topheight}
\setlength{\topheight}{70pt}
\addtolength{\topheight}{\headheight}
\DeclareLayer[
background,
contents={%
	\color{PrimaryColor}%
	\rule{\paperwidth}{\topheight}%
}%
]{scrheadings.head.background}
\makeatother

%Adding the Layer to the pagestyles
\AddLayersAtBeginOfPageStyle{scrheadings}{%
	scrheadings.head.background,%
}
\AddLayersAtBeginOfPageStyle{plain.scrheadings}{%
	scrheadings.head.background}


% Pseudocode
\usepackage{algorithm}
\usepackage{algorithmic}

% Code Listings
\usepackage{listings}
\usepackage{color}
\usepackage{beramono}

\definecolor{DarkPurple}{rgb}{0.4, 0.1, 0.4}
\definecolor{DarkCyan}{rgb}{0.0, 0.5, 0.4}
\definecolor{LightLime}{rgb}{0.3, 0.5, 0.4}
\definecolor{Blue}{rgb}{0.0, 0.0, 1.0}

% Page colors
\definecolor{PrimaryColor}{HTML}{CDDC39}
\definecolor{SecondaryColor}{HTML}{374046}

\usepackage{afterpage}
\usepackage{xcolor}
\usepackage{sectsty}

\color{SecondaryColor}

\lstdefinestyle{eclipse-style}{
	language=Java,
	columns=flexible,
	showstringspaces=false,
	basicstyle=\footnotesize\ttfamily, 
	keywordstyle=\bfseries\color{DarkPurple},
	commentstyle=\color{LightLime},
	stringstyle=\color{Blue}, 
	escapeinside={£}{£}, % latex scope within code
	morekeywords={length},
	numbers=left,
	numberstyle=\tiny\color{black},
	frame=single,
}
\lstset{style=eclipse-style}


%risk-rating
\newcommand\risk[2]{
	\begin{tikzpicture}
	\draw [thick, |->] (0,2) -- (#2,2);
	\draw [fill=PrimaryColor, thick] (#1,2) circle [radius=0.2];
	\end{tikzpicture}
}

% Theorems \begin{mytheo}{title}{label}
\usepackage{tcolorbox}
\tcbuselibrary{theorems}
\newtcbtheorem[number within=section]{definiton}{Definition}%
{fonttitle=\bfseries}{def}
\newtcbtheorem[number within=section]{remember}{Merke}%
{fonttitle=\bfseries}{rem}
\newtcbtheorem[number within=section]{hint}{Hinweis}%
{fonttitle=\bfseries}{hnt}